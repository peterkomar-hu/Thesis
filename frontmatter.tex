%% frontmatter.tex
%%

\title{Systems and Protocols for Quantum Computation and Quantum Metrology}
\author{P\'{e}ter K\'{o}m\'{a}r}
\degreemonth{December} % month final submission occurs.
\degreeyear{2015}
\degree{Doctor of Philosophy}
\field{Physics} 
\department{Physics}
\advisor{Mikhail D. Lukin} % Category I added.

\maketitle
\copyrightpage
  

\begin{abstract} 
% limited to 1.5 pages, double-spaced (Registrar's Office guidelines).
% Also limited to 350 words. I claim $\mu \sim \omega^4$ is a single word.
   
% Abstract about 
% \begin{itemize}
%   \item Quantum computation protocols
%   \item and their use
%   \item Quantum metrology protocols
%   \item and their use
%   \item systems capable of realizing them
%   \item namely: optomechanical systems
%   \item atomic sytems
% \end{itemize} 

The current frontier of our understanding of the physical universe is dominated
by quantum phenomena. Uncovering the prospects and limitations of acquiring and
processing information in this counter-intuitive realm is one of the greatest
challenges of our times. This thesis presents the analysis of several new
model systems and protocols for quantum computation and metrology.

First, we analyze a member of a class of quantum optomechanical systems,
nano-fabricated devices exhibiting quantum phenomena in both optical and
mechanical degrees of freedom. We investigate the strength of non-classical
correlations in a model system of two optical and one mechanical mode.  Later,
we propose and analyze experimental protocols that exploit these correlations in
order to do quantum computation.

We then turn our attention to atom-cavity systems, and investigate the
possibility of using them as robust information storage and relay nodes.
We present a minimalistic scheme for the two-qubit quantum gate with inherent
error-reduction capabilities. Later, we consider several promising remote
entangling protocols employing this robust gate, and we use this as a testing
ground to shed light on the performance of the gate in a real application.

Finally, we present a new protocol for running multiple, remote atomic clocks in
quantum unison. We show that by creating a cascade of independent GHZ states 
distributed across the network, the scheme asymptotically reaches the
Heisenberg limit, the fundamental limit of accuracy. We propose an experimental
realization of such a network consisting of neutral atom clocks, and provide an
estimate on how much experimental imperfections would limit its precision. 




\end{abstract}



\newpage
\addcontentsline{toc}{section}{Table of Contents}
\tableofcontents

% these are optional in the Jan 2000 Harvard thesis GSAS guide:
\listoffigures
% use \caption[lst-entry]{Text of table caption} to define the title of the
% figure
\listoftables


% cccccccccccccccccccccccccccccccccccccccccccccccccccccccccccccccccccccccccc
\begin{citations}

\vspace{0.8in}

\ssp
\noindent
Most of the chapters of this thesis have appeared in print elsewhere. By
chapter number, they are:
\begin{itemize}
  	\item 
		Chapter \ref{ch:Komar2013}: 
		``Single-photon nonlinearities in two-mode optomechanics,'' 
		P. K\'{o}m\'{a}r,
		S. D. Bennett, 
		K. Stannigel,  
		S. J. M. Habraken,  
		P. Rabl,  
		P. Zoller, 
		and 
		M. D. Lukin, 
		\emph{Phys.	Rev.  A} 
		\textbf{87}, 
		013839 
		(2013).
  	\item 
		Chapter \ref{ch:Stannigel2012}: 
		``Optomechanical Quantum Information Processing with Photons and Phonons,'' 
		K. Stannigel,  
		P. K\'{o}m\'{a}r, 
		S. J. M. Habraken,  
		S. D. Bennett,
		M. D. Lukin,
		P. Zoller,   
		and P. Rabl, 
		\emph{Phys. Rev. Lett.} 
		\textbf{109},
		013603 (2012).
  	\item 
		Chapter \ref{ch:Borregaard_PRL2015}: 
		``Heralded Quantum Gates with Integrated Error Detection in Optical
		Cavities,'' 
		J. Borregaard, 
		P. K\'{o}m\'{a}r, 
		E. M. Kessler, 
		A. S. S{\o}rensen, 
		and M. D. Lukin,
 		\emph{Phys. Rev. Lett.} 
		\textbf{114},
		110502 (2015).
  	\item 
		Chapter \ref{ch:Borregaard_PRA2015}: 
		``Long-distance entanglement distribution using individual atoms in optical
		cavities,'' 
		J. Borregaard, 
		P. K\'{o}m\'{a}r, 
		E. M. Kessler, 
		A. S. S{\o}rensen, 
		and M. D. Lukin, 
 		\emph{Phys. Rev. A} 
		\textbf{92},
		012307 (2015).
  	\item 
		Chapter \ref{ch:Kessler2014}: 
		``Heisenberg-Limited Atom Clocks Based on Entangled Qubits,''
		E. M. Kessler, 
		P. K\'{o}m\'{a}r,  
		M. Bishof, 
		L. Jiang, 
		A. S. S{\o}rensen, 
		J. Ye, 
		and M. D. Lukin,
 		\emph{Phys. Rev. Lett.} 
		\textbf{112},
		190403 (2014).
  	\item 
		Chapter \ref{ch:Komar2014}: 
		``A quantum network of clocks,''
		P. K\'{o}m\'{a}r,  
		E. M. Kessler, 
		M. Bishof, 
		L. Jiang, 
		A. S. S{\o}rensen, 
		J. Ye, 
		and M. D. Lukin,
 		\emph{Nature Physics} 
		\textbf{10},
		582�587 (2014).
	\item 
		Chapter \ref{ch:Komar2015}: 
		``\warn{?}Quantum network of netural atom clocks,''
		P. K\'{o}m\'{a}r,  
		T. Topcu, 
		E. M. Kessler, 
		A. Derevianko, 
		A. S. S{\o}rensen, 
		and M. D. Lukin,
 		\emph{\warn{??}} 
		\textbf{\warn{??}},
		\warn{??} (2015\warn{?}).			 
\end{itemize}

\end{citations}




\begin{acknowledgments}

First of all, I would like to thank my research advisor, Prof. Mikhail Lukin,
for his scientific insights and ideas, and especially for his attention and
patience in guiding my work and education.

I would like to thank other members of my thesis committee, Prof. John Doyle
and Prof. Subir Sachdev, with whom I was fortunate to work
together as a teaching fellow. Their knowledge and thoroughness
inspired both my research and teaching.

I am grateful to Prof. Andrei Derevianko at University of Nevada, Prof.
Pierre Meystre at University of Arizona, Prof. Peter Rabl in Vienna, Till
Rosenband at Harvard, Prof. Anders S{\o}rensen in Copenhagen, Prof. Vladan
Vuleti\v{c} at MIT, Prof. Jun Ye at JILA, and Prof. Peter Zoller in Innsbruck
for the enlightening discussions and their invaluable contributions to my
research in the past five years.

I would like to thank all colleagues with whom I worked namely, Michael Bishof,
Soonwon Choi, Manuel Endres, Ruffin Evans, Michael Goldman, Michael Gullans,
Steven Habraken, M\'{a}rton Kan\'{a}sz-Nagy, Ronen Kroeze, Peter Maurer, Travis
Nicholson, J\'{a}nos Perczel, Thibault Peyronel, Arthur Safira, Alp Sipahigil,
Kai Stannigel, Alex Sushkov, Jeff Thompson, Turker Topcu, Dominik Wild, Norman
Yao, Leo  Zhou. I am especially grateful to Steven Bennett, Johannes Borregaard,
and Eric Kessler; besides many years of fruitful collaboration, they helped me
as mentors and friends.

I would also like to thank David Morin, Jacob Barandes, Nick Schade and people
from the Bok Center, John Girash, Colleen Noonan, and Matthew Sussman, for
their efforts in guiding me to become a better teacher.

I am thankful for my friends in Cambridge and Boston: Travis and John Woolcott,
who gave me tremendous help during my first year, and continued to keep an eye
on me; Bence B\'{e}ky and Gitta Szabari for teaching me the tricks and
traditions of living in the US; and Kartiek Agarwal, Debanjan Chowdhury and Ilya
Feige for our endless discussions about life.

The Physics Department staff has been an invaluable resource.
I would like to thank
Monika Bankowski, Jennifer Bastin, Lisa Cacciabaudo, Karl Coleman, Carol Davis,
Sheila Ferguson, Joan Hamilton, Dayle Maynard, Clare Ploucha, Janet Ragusa and
Sarah Roberts, for helping me at countless occasions.

I am grateful to the Harvard International Office, and especially to Darryl
Zeigler, for all the help making me feel myself at home at Harvard.

I am thankful to the Office of Career Services, and especially Laura Stark and
Heather Law, for helping me transition to the next stage of my career.

Finally, I would like to thank my family, Erzs\'{e}bet K\'{o}m\'{a}r, Antal
K\'{o}m\'{a}r, Anna K\'{o}m\'{a}r and Szilvia Kiriakov, for their immense
support and understanding towards my education and work. I cannot thank you
enough. This thesis is dedicated to you.
 






 




\end{acknowledgments}





%ddddddddddddddddddddddddddddddddddddddddddddddddddddddddddddddddddddddddddd
\dedication

\begin{quote}
\hsp
\em
\raggedleft

Dedicated to my parents Erzs\'{e}bet and Antal,\\
my sister Anna,\\
and my financ\'{e}e Szilvia.

\end{quote}


\newpage

\startarabicpagination

%%% end

