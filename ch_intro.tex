\chapter{Introduction and Motivation}
%%%%%%%%%%%%%%%%%%%%%%%%%%%%%%%%%%%%%%%%%%%%%%%%

\section{Overview and Structure}
The field of quantum science aims to answer conceptual and practical questions
about the fundamental behavior, controllability and applicability of 
systems governed by quantum mechanics. Such systems arise whenever a few 
degrees of freedom of a physical system become isolated from their macroscopic
surroundings. 

Realizing and maintaining the required isolation is a formidable
task. The interaction within the isolated components needs to be
much stronger than the collective coupling to modes of the environment.
Once achieved, the system starts to manifestly explore an expanded set of states: Its
dynamics is not constrained to a countable number of pointer (or
``classical'') states anymore, originally selected by the environment, and it
moves around smoothly in the entire Hilbert space, with its motion governed by the
Hamilton operator. Internal components of such a system are said to be ``strongly
coupled'', and their dynamics ``coherent''.

The variety of (``quantum'') states in the Hilbert space  gives rise
to counter-intuitive phenomena such as superposition, tunneling and
entanglement.
Besides being academically exciting, these phenomena hold the promise that
future devices and protocols relying on them will perform better than any
conceivable scheme based on solely classical dynamics. 

The discovery of
efficient quantum algorithms for problems that are conjectured to be not
efficiently computable fueled the field of Quantum Computing. In Chapter
\ref{ch:Komar2013} and \ref{ch:Stannigel2012}, we analyze the
capabilities of nano-scale optomechanical systems to perform coherent logical
operations, the elemental steps of quantum computation.

Protocols that
rely on entanglement to distribute secret keys between distant parties with
perfect security, which relies on fundamental physical limits rather than
computational complexity, is the main focus of Quantum Communication. In Chapter
\ref{ch:Borregaard_PRL2015} and \ref{ch:Borregaard_PRA2015}, we describe how a system consisting of a few atoms
isolated in an optical cavity can be used to realize a quantum gate with
integrated error-detection, and we analyze its usefulness in a quantum
communication setting.

The idea of preparing a detector in quantum
superposition in order to focus its sensitivity to the quantity of interest is
the central topic of Quantum Metrology. In Chapter \ref{ch:Kessler2014},
\ref{ch:Komar2014} and \ref{ch:Komar2015}, we present a
protocol for operating a network of atomic clocks, which combines local and
remote entanglement to surpass the accuracy of classical protocols and
asymptotically reach the fundamental quantum limit of precision, the Heisenberg limit. 


 
 
 
 
 
 
\section{Optomechanical Systems}
The current fabrication technology allows the creation of chips with
nanometer-scale features such as waveguides \cite{Mekis1996}, photonic band-gap
materials \cite{Foresi1997}, non-linear inductive elements \cite{Makhlin1999},
antenna arrays \cite{Yu2014}, optical cavities \cite{Painter2001}, and
mechanical resonators of various shapes and sizes \cite{Aspelmeyer2014}.

Coupling components with different physical nature can give rise to devices
incorporating the best characteristics of each component.
Optical components are fast ($\sim 100\,\mathrm{THz}$) and fairly isolated,
however making them interact with each other strongly is challenging
\cite{Chang2007}.
Mechanical components, on the other hand, are much slower ($\sim
10-100\,\mathrm{GHz}$), but usually much more sensitive to changes in their
surroundings \cite{Aspelmeyer2014}. Using mechanical elements as transducers
between two optical degrees of freedom allows efficient filters and frequency
converters \cite{Eichenfield2009} when driven by classical light.

The interaction between the light field and a mechanical surface, originates
mainly from the light-pressure displacing the mechanics, giving rise to a
non-linear parametric coupling between light intensity and mechanical motion
\cite{Meystre2013}. In Chapter \ref{ch:Komar2013}, we
analyze a quantum model of two optical cavities coupled and a mechanical
oscillator coupled by this coupling. We find that if the system is driven by a
weak laser pulse (accurately described by a Poisson-process of photon
arrivals), the output light exhibits super- and sub-Poissionan characteristics.
We found that if the system is properly tuned, such that the energy difference between
the two optical modes is bridged by the mechanical mode, and it is sufficiently
cooled down, such that it is in its ground state the majority of the time,
then we can use the photons leaving the output port to herald the creation of a
single mechanical excitation.

Using oscillators as quantum registers in a future quantum computer requires
them to be anharmonic \cite{Majer2007}. Consecutive levels of an
anharmonic oscillator are separated by unequal frequency intervals, which makes
it possible to address transitions independently. The inherent
non-linearity of the optomechanical coupling holds the promise of rendering the
coupled oscillator system sufficiently anharmonic for computational tasks. In
Chapter \ref{ch:Stannigel2012}, we propose a quantum logic architecture based on
coupled optomechanical components. We find that with sufficiently strong cooling
the system exhibits non-classical behaviors and able to store information and
perform logic gates on the qubits.






\section{Atom-Cavity Systems}
Coupling individual atoms to optical cavities is one of the most effective ways
to realize a well-controllable and manifestly quantum system \cite{Mabuchi2002,
Walther2006}.
The first model, named after E. Jaynes and F. Cummings \cite{Jaynes1963,
Shore1993}, describes the coherent dynamics of a single transition between two
levels of an atom and a single, confined optical mode. This model is exactly
solvable and serves as a great source of intuition.

As the physical size of an optical cavity is decreased, the zero-point
electric field corresponding to the ground state of its modes increases.
As a result, individual atoms placed inside such a cavity, coupled through their
electric dipole moment, start to interact strongly with the optical modes. Once
this interaction becomes much stronger than the atoms coupling to the radiation
environment outside of the cavity, the system becomes strongly coupled, and
their states hybridize. From a spectroscopic point of view, this results in a
resolvable splitting of the optical resonances.

Optical cavities built on photonic crystal waveguides \cite{Tiecke} have
high zero-point electric fields, and produce couplings
to atoms larger than their spontaneous emission rate. Their
observed lifetime is then considerably decreased \cite{Englund2005}; an effect
called Purcell-enhancement.

Once such strong interaction is demonstrated, the prospect of using these
systems for coherent quantum logic operations becomes realistic. In Chapter
\ref{ch:Borregaard_PRL2015}, we consider a model of three atoms placed inside
and coupled by a single optical cavity. We show that, with tailored driving pulses
and proper choice of atomic levels, this system can perform a controlled-NOT
operation on two of the atoms can be made resistant to the dominating error,
caused by the loss of a photon, by post-selecting on the third atom's
state.

\section{Quantum Repeaters}
Creating quantum entanglement between systems separated by
large distances is the most important prerequisite of quantum communication
protocols. Experimental realizations rely on exchanging weak light pulses
via carefully monitored optical fibers \cite{Peev2009}. The
reliability of direct transmission of photon pulses is limited by the
fiber attenuation length, the maximum of which ($\sim 20\,\mathrm{km}$) is
achieved at telecom wavelength ($\sim 1\,\mu\mathrm{m}$).

Classical communication solves the attenuation problem by incorporating fiber
segments which amplify the signal. Conceptually, this classical amplification
relies on detecting some of the signal's photons and emitting more in synchrony.
Unfortunately, such a process fails for quantum channels using single-photon
pulses, because the detection event can measure the photon only once,
therefore it necessarily discards essential information about its quantum state.

Overcoming the attenuation problem in quantum channels requires using more
resources. The scheme of quantum repeaters consists of repeater stations placed
between the sender and the receiver \cite{bennett2, bennett, duan3}.
These stations, instead of relaying the information forwards, create pairwise
entanglement with their neighbors using direct photon transmission via the
fibers which are much shorter than the total length. Once all entangled pairs
are confirmed, each station performs a local quantum logic operation between the two
pairs they have access to. This is called entanglement connection, resulting in
entanglement between the two outermost parties. An alternative solution 
encodes the information in states of a many-photon pulse, and apply periodic
quantum error-correction along its way \cite{Muralidharan2015}.

The reliability and transmission rate of such a quantum repeater protocol
depends strongly on the fidelity of the entanglement connection step. In Chapter
\ref{ch:Borregaard_PRA2015}, we show that the atom-cavity system described in
Chapter \ref{ch:Borregaard_PRL2015}, used to perform the quantum logic gate,
performs outstandingly for total distances $\sim
100-1000\,\mathrm{km}$.



\section{Atomic Clocks and Quantum Metrology}
Atomic clocks are the currently available best time-keeping devices. They are
used to keep and, in many cases, to broadcast a precise time and frequency
standard. The workings of atomic clocks rely on two main components: the
reference oscillator, realized by an isolated, narrow-linewidth electromagnetic
transition of an atomic species \cite{Derevianko2011}; and the slaved
oscillator, the microwave or optical source of strong, coherent field. The
clock keeps time by periodically interrogating the atoms with the field of the
slaved oscillator (laser), and measuring the deviation of their frequencies.
Then, the measurement result is  used to correct the laser's frequency. This
closes the feedback loop, and results in an actively stabilized laser field, which serves
as a clock signal \cite{Diddams2004}.

The accuracy of an atomic clock, characterized by the average fractional
frequency deviation or Allan-deviation 
\cite{Allan1966, Rutman1978}, is determined by several factors. Employing
higher reference frequency, longer interrogation cycle time, longer averaging
time and more atoms all improve the overall accuracy. Consequently, there are
many ways to improve the accuracy. Choosing an atomic transition with optical
frequency, instead of microwave, boosts the clock's performance by five orders
of magnitude. The central frequency of current record-holder atomic clocks are all
in the optical domain \cite{Ludlow2015}. The optimal interrogation time usually
falls slightly above the coherence time of the laser, and fails to improve the scheme
beyond atomic coherence time even in schemes that eliminate the limiting effect
of the laser \cite{Borregaard2013, Rosenband2013}. The maximal
total averaging time is usually determined by the refresh rate of the clock
signal required by the application, and by other noises such as frequency
flicker noise \cite{Barnes1966}.

The dependence of the precision on the number of atoms is a true quantum
phenomenon.
It is due to the fundamental limit on the information a single measurement can
obtain about any atom. When $N$ atoms are measured independently, the
Allen-deviation scales as $\propto N^{-1/2}$, and is said to be limited by
``projection noise'' or the ``Standard Quantum Limit''. It originates from the
inherent uncertainty of the measurement result of a single two-level system
prepared in superposition \cite{Santarelli1998}.

The Standard Quantum Limit describes the limit of precision if all atoms are
prepared and measured independently, or as an uncorrelated ensemble, but it is
higher than the fundamental quantum limit, due to Heisenberg uncertainty, which
predicts $\propto N^{-1}$ scaling of the precision with increasing atom number
$N$. This gap between the two limits, and proposals trying to close it, is the
central topic of Quantum Metrology \cite{Giovanetti2011, Escher:2011fn}. By
preparing the collection of atoms in an entangled state, the subsequent
measurement will have lower uncertainty and provide more information about the
detuning of the laser frequency form the atomic reference. States such as
squeezed states \cite{Andre2004, Borregaard2013_nearHeisenberg},
Greenberger-Horne-Zeilinger (GHZ) states \cite{Wineland1998, Bollinger1996}, and
optimally entangled states \cite{Buzek1999, Berry2009}, all promise a
significant improvement, and some even reach the
Heisenberg limit for large $N$. In Chapter \ref{ch:Kessler2014}, we calculate
a limit on the best achievable performance using a cascade of GHZ states, and
compare it with the other algorithms. We find that for total averaging
times shorter than the atomic coherence time, the precision of our scheme
surpasses the precision of the best classical protocol.

Once we establish that entangling the available atoms is beneficial, finding the
optimal protocol for a network of atomic clocks becomes an important problem. In
Chapter \ref{ch:Komar2014}, we assume that a large number of identical atomic
clocks are joined together in a quantum network using a quantum communication
scheme. We show that this network can be operated in a way that every clock
atom is entangled with all atoms in the network, forming a global
GHZ state. We characterize the enhancement of the overall precision and compare
it with precisions of schemes using only local or no entanglement.





\section{Rydberg Blockade}
In most cases, interactions between atoms in cold ensembles are accurately
modeled by short-range or contact interactions \cite{Cheng2010}.
They can be neglected if the gas is not too dense, and the duration of the
phenomena under investigation is short compared to the inverse of average
collision rate. This breaks down when a few atoms acquire significant magnetic
or electric dipole moments and, as a result, start interacting via dipole-dipole
interaction, whose strength scales as $\propto R^{-3}$ with separation $R$.

One way to induce a strong electric dipole moment in an atom is to optically
excite the outermost electron to a level with high principle quantum number ($n
\gtrsim 30$), a Rydberg level. Rydberg levels have small energy spacing
($\propto n^{-3}$), long spontaneous lifetimes ($\propto n^3$), and strong
transition dipole moments ($\propto n^2$) \cite{Saffman2010}. These properties
make Rydberg atoms a promising tool to realize fast and reliable quantum
logic operations \cite{Lukin2001}.  

Blockade between Rydberg atoms is an especially strong and promising phenomenon.
When one atom gets excited into a Rydberg state, the long-range interaction
originating from its (transition) electric dipole moment shifts the Rydberg
levels of the other atoms out of resonance, and prevents them from being excited
\cite{Urban2009}. This effect creates an exceptionally strong non-linearity: it
limits the number of Rydberg excitations in the cloud to zero and one, allowing
the cloud to be used as a qubit. In Chapter \ref{ch:Komar2015}, we present and
analyze a quantum protocol that relies on strong Rydberg blockade to perform
fast, high-fidelity operations between different quantum registers. We propose
to use different delocalized spin-waves of the atomic cloud to store
information, and employing the Rydberg blockade to mediate interactions between
them. We show that even after taking the physical imperfections into account,
our scheme provides a feasible way to prepare the network-wide global GHZ state
required by the quantum clock network of Chapter \ref{ch:Komar2014}.

 









