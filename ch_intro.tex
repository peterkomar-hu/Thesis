\chapter{Introduction and Motivation}
%%%%%%%%%%%%%%%%%%%%%%%%%%%%%%%%%%%%%%%%%%%%%%%%

\section{Overview and Structure}
The field of quantum science aims to answer conceptual and practical questions
about the fundamental behavior, controllability and applicability of 
systems governed by quantum mechanics. Such systems arise once a few 
degrees of freedom of a physical system become isolated from their macroscopic
surroundings. 

Realizing and maintaining the required isolation is a formidable
task. The interaction within the isolated components needs to be
much stronger than the collective coupling to modes of the environment.
Once achieved, the system starts to manifestly explore an expanded set of states: Its
dynamics is not constrained to a countable number of pointer (or
``classical'') states anymore, originally selected by the environment, and it
moves around smoothly in the entire Hilbert space, with its motion governed by the
Hamilton operator. Internal components of such a system are said to be ``strongly
coupled'', and their dynamics ``coherent''.

The variety of (``quantum'') states in the Hilbert space  gives rise
to counter-intuitive phenomena such as superposition, tunnelling and
entanglement.
Besides being academically exciting, these phenomena hold the promise that
future devices and protocols relying on them will perform better than any
concievable scheme based on solely classical dymanics. 

The discovery of
efficient quantum algorithms for problems that are conjectured to be not
efficiently computable fuelled the field of Quantum Computing. In Chapter
\ref{ch:Komar2013} and \ref{ch:Stannigel2012}, we analyze the
capabilities of nano-scale optomechanical systems to perform coherent logical
operations, the elemental steps of quantum computation.

Protocols that
rely on entanglement to distribute secret keys between distant parties with
perfect security, which relies on fundamental physical limits rather than
computational complexity, is the main focus of Quantum Communication. In Chapter
\ref{ch:Borregaard_PRL2015} and \ref{ch:Borregaard_PRA2015}, we describe how a system consisting of a few atoms
isolated in an optical cavity can be used to realize a quantum gate with
integrated error-detection, and we analyze its usefulness in a quantum
communitation setting.

The idea of preparing a detector in quantum
superposition in order to focus its sensitivity to the quantity of interest is
the central topic of Quantum Metrology. In Chapter \ref{ch:Kessler2014},
\ref{ch:Komar2014} and \ref{ch:Komar2015}, we present a
protocol for operating a network of atomic clocks, which combines local and
remote entanglement to surpass the accuracy of classical protocols and asymtotically
reach the fundamental quantum limit of precision, the Heisenberg limit. 


 
 
 
 
 
 
\section{Optomechanical Systems}
The current fabrication technology allows the creation of chips with
nanometer-scale features such as waveguides [REF:waveguide], photonic band-gap
materials [REF:photonic crystals], micron-size capacitive and inductive
components [REF: LC], antenna arrays [REF: antennas], optical cavities [REF:
optical cavities], and mechanical resonators of various shapes and sizes [REF:
mechanical resonator]. 

Engineering coupling between components with different physical nature can give
rise to devices incorporating the best characteristics of each. Optical
components are fast ($\sim 100\,\mathrm{THz}$) and fairly isolated, however the
interaction between optical components is usually tiny [REF: Kerr effect on
chips]. Mechanical components, however, are much slower ($\sim
10-100\,\mathrm{GHz}$), but usually much more sensitive to changes in their
surroundings. Using mechanical elements as transducers between two optical
degrees of freedom allows efficient filters and frequency converters [REF:
Painter: frequency conversion] when driven by classical light.

The interaction between the light field and a mechanical surface, originates
mainly from the light-pressure displacing the mechanics, giving rise to a
non-linear parametric coupling between light intensity and mechanical motion
[REF: Pierre Meyste's calculation]. In Chapter \ref{ch:Komar2013}, we
analyze a quantum model of two optical cavities coupled by a mechanical
oscillator. We find that if the system is driven by a weak laser pulse
(accurately discribed by a Poission process of photon arrivals), the output
light exhibits super- and sub-Poissionan charateristics. We found that if the
system is properly tuned, such that the energy difference between
the two optical modes is bridged by the mechanical mode, and it sufficiently
cooled down, such that it is in its ground state the majority of the time,
then we can use the photons leaving the output port to herald the creation of a
single mechanical excitation.

Using oscillators as quantum registers in a future quantum computer requires
them to be anharmonic [REF: Superconducting QC]. Consecutive levels of an
anharmonic oscillator are separated by unequal frequency intervals, which makes
it possible to address certain transitions individually. The inherent
non-linearity of the optomechanical coupling holds the promise of rendering the
coupled oscillator system sufficiently anharmonic for this task. In Chapter
\ref{ch:Stannigel2012}, we propose an quantum logic architecture based on
coupled optomechanical components. We find that with sufficiently strong cooling
the system exhibits non-classical behaviors and able to store and perform
quantum logic gates on qubits of information.






\section{Atom-Cavity Systems}

\section{Rydberg Interactions}

\section{Quantum Repeaters}

\section{Atomic Clocks and Quantum Metrology}