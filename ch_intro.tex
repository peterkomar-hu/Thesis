\chapter{Introduction and Motivation}
%%%%%%%%%%%%%%%%%%%%%%%%%%%%%%%%%%%%%%%%%%%%%%%%

\section{Overview and Structure}
The field of quantum science aims to answer conceptual and practical questions
about the fundamental behavior, controllability and applicability of physical
systems governed by quantum mechanics. Such systems arise once a few 
degrees of freedom of a physical system become isolated from their macroscopic
surroundings. 

Realizing and maintaining the required isolation is a formidable
task. The interaction within the isolated components needs to be
much stronger than the collective coupling to modes of the environment.
Once achieved, the system starts to manifestly explore an expanded set of states: Its
dynamics is not constrained to a countable number of pointer (or
``classical'') states anymore, originally selected by the environment, and it
moves around smoothly in the entire Hilbert space, with its motion governed by the
Hamilton operator. Internal components of such a system are said to be ``strongly
coupled'', and their dyamnics ``coherent''.

The variety of (``quantum'') states in the Hilbert space  gives rise
to counter-intuitive phenomena such as superposition, tunnelling and
entanglement.
Besides being academically exciting, these phenomena hold the promise that
future devices and protocols relying on them will perform better than any
concievable scheme based on solely classical dymanics. 

The discovery of
efficient quantum algorithms for problems that are conjectured to be not
efficiently computable fuelled the field of Quantum Computing. In Chapter
\ref{ch:Komar2013} and \ref{ch:Stannigel2012}, we present the
capabilities of nano-scale optomechanical systems to perform coherent logical
operations, the elemental steps of quantum computation.

Protocols that
rely on entanglement to distribute secret keys between distant parties with
asymptotically perfect security is the main focus of Quantum
Communication. In Chapter \ref{ch:Borregaard_PRL2015} and 
\ref{ch:Borregaard_PRA2015}, we describe how a system consisting of a few atoms
isolated in an optical cavity can be used to realize a quantum gate with
integrated error-detection, and we analyze its usefulness in a quantum
communitation setting.

The idea of preparing a detector in quantum
superposition in order to focus its sensitivity to the quantity of interest is
the central topic of Quantum Metrology. In Chapter \ref{ch:Kessler2014},
\ref{ch:Komar2014} and \ref{ch:Komar2015}, we present an operational
protocol for a network of atomic clocks, which combines local and remote
entanglement to surpass the accuracy of classical protocols and asymtotically
reach the fundamental quantum limit of precision, the Heisenberg limit. 


 
 
 
 
 
 
\section{Optomechanical Systems}

\section{Atom-Cavity Systems}

\section{Rydberg Interactions}

\section{Quantum Repeaters}

\section{Atomic Clocks and Quantum Metrology}